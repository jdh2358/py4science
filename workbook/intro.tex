This document contains a set of small problems, drawn from many different
fields, meant to illustrate commonly useful techniques for using Python
in scientific computing.

All problems are presented in a similar fashion: the task is explained
including any necessary mathematical background and a `code skeleton'
is provided that is meant to serve as a starting point for the solution
of the exercise. In some cases, some example output of the expected
solution, figures or additional hints may be provided as well. 

The accompanying source download for this workbook contains the complete
solutions, which are not part of this document for the sake of brevity.

For several examples, the provided skeleton contains pre-written tests
which validate the correctness of the expected answers. When you have
completed the exercise successfully, you should be able to run it
from within IPython and see something like this (illustrated using
a trapezoidal rule problem, whose solution is in the file \texttt{trapezoid.py}):

\begin{lstlisting}
In [7]: run trapezoid.py
....
----------------------------------------------------------------------
Ran 4 tests in 0.003s

OK
\end{lstlisting}

This message tells you that 4 automatic tests were successfully executed.
The idea of including automatic tests in your code is a common one
in modern software development, and Python includes in its standard
library two modules for automatic testing, with slightly different
functionality: \texttt{unittest} and \texttt{doctest}. These tests
were written using the \texttt{unittest} system, whose complete documentation
can be found here: \url{http://docs.python.org/lib/module-unittest.html}. 

Other exercises will illustrate the use of the \texttt{doctest} system,
since it provides complementary functionality.
