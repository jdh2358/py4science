\texttt{R}, a statistical package based on \texttt{S}, is viewd by some
as the best statistical software on the planet, and in the open source
world it is the clear choice for sophisticated statistical analysis.
Like python, \texttt{R} is an interpreted language written in C with an
interactive shell.  Unlike python, which is a general purpose
programming language, \texttt{R} is a specialized statistical language.
Since python is a excellent glue language, with facilities for
providing a transparent interface to FORTRAN, C, C++ and other
languages, it should come as no surprise that you can harness
\texttt{R}'s immense statistical power from python, through the
\texttt{rpy} third part extension library.

However, \texttt{R} is not without its warts.  As a language, it lacks
python's elegance and advanced programming constructs and idioms.  It
is also GPL, which means you cannot distribute code based upon it
unhindered: the code you distribute must be GPL as well (python, and
the core scientific extension libraries, carry a more permissive
license which support distribution in closed source, proprietary
application).

Fortunately, the core tools scientific libraries for python (primarily
\texttt{numpy} and \texttt{scipy.stats}) provide a wide array of
statistical tools, from basic descriptive statistics (mean, variance,
skew, kurtosis, correlation, \dots) to hypothesis testing (t-tests,
$\chi$-Square, analysis of variance, general linear models, \dots) to
analytical and numerical tools for working with almost every discrete
and continuous statistical distribution you can think of (normal,
gamma, poisson, weibull, lognormal, levy stable, \dots).



