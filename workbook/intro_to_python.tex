
\chapter[Python intro]{A whirlwind tour of python and the standard library}

This is a quick-and-dirty introduction to the python language for
the impatient scientist. There are many top notch, comprehensive introductions
and tutorials for python. For absolute beginners, there is the \textit{Python
Beginner's Guide}.%
\footnote{http://www.python.org/moin/BeginnersGuide%
} The official \textit{Python Tutorial} can be read online%
\footnote{http://docs.python.org/tut/tut.html%
} or downloaded%
\footnote{http://docs.python.org/download.html%
} in a variety of formats. There are over 100 python tutorials collected
online.%
\footnote{http://www.awaretek.com/tutorials.html%
}

There are also many excellent books. Targetting newbies is Mark Pilgrim's
\textit{Dive into Python} which in available in print and for free
online%
\footnote{http://diveintopython.org/toc/index.html%
}, though for absolute newbies even this may be too hard \cite{Dive}.
For experienced programmers, David Beasley's \textit{Python Essential
Reference} is an excellent introduction to python, but is a bit dated
since it only covers python2.1 \cite{Beasley}. Likwise Alex Martelli's
\textit{Python in a Nutshell} is highly regarded and a bit more current
-- a 2nd edition is in the works\cite{Nutshell}. And \textit{The
Python Cookbook} is an extremely useful collection of python idioms,
tips and tricks \cite{Cookbook}.

But the typical scientist I encounter wants to solve a specific problem,
eg, to make a certain kind of graph, to numerically integrate an equation,
or to fit some data to a parametric model, and doesn't have the time
or interest to read several books or tutorials to get what they want.
This guide is for them: a short overview of the language to help them
get to what they want as quickly as possible. We get to advanced material
pretty quickly, so it may be touch sledding if you are a python newbie.
Take in what you can, and if you start getting dizzy, skip ahead to
the next section; you can always come back to absorb more detail later,
after you get your real work done.


\section{Hello Python}

Python is a dynamically typed, object oriented, interpreted language.
Interpreted means that your program interacts with the python interpreter,
similar to Matlab, Perl, Tcl and Java, and unlike FORTRAN, C, or C++
which are compiled. So let's fire up the python interpreter and get
started. I'm not going to cover installing python -- it's standard
on most linux boxes and for windows there is a friendly GUI installer.
To run the python interpreter, on windows, you can click \texttt{Start->All
Programs->Python 2.4->Python (command line)} or better yet, install
\texttt{ipython}, a python shell on steroids, and use that. On linux
/ unix systems, you just need to type \texttt{python} or \texttt{ipython}
at the command line. The \texttt{>\,{}>\,{}>} is the default python
shell prompt, so don't type it in the examples below

\begin{lyxcode}
>\,{}>\,{}>~print~'hello~world'

hello~world


\end{lyxcode}
As this example shows, \textit{hello world} in python is pretty easy
-- one common phrase you hear in the python community is that {}``it
fits your brain''. -- the basic idea is that coding in python feels
natural. Compare python's version with \textit{hello world} in C++

\begin{lyxcode}
//~C++

\#include~<iostream>

int~main~()

\{~~~

~~std::cout~<\,{}<~\char`\"{}Hello~World\char`\"{}~<\,{}<~std::endl;

~~return~0;

\}
\end{lyxcode}

\section[Calculator]{\label{sec:into_calculator}Python is a calculator}

Aside from my daughter's solar powered cash-register calculator, Python
is the only calculator I use. From the python shell, you can type
arbitrary arithmetic expressions.

\begin{lyxcode}
>\,{}>\,{}>~2+2

4

>\,{}>\,{}>~2{*}{*}10

1024

>\,{}>\,{}>~10/5

2

>\,{}>\,{}>~2+(24.3~+~.9)/.24

107.0

>\,{}>\,{}>~2/3

0
\end{lyxcode}
The last line is a standard newbie gotcha -- if both the left and
right operands are integers, python returns an integer. To do floating
point division, make sure at least one of the numbers is a float

\begin{lyxcode}
>\,{}>\,{}>~2.0/3

0.66666666666666663
\end{lyxcode}
The distinction between integer and floating point division is a common
source of frustration among newbies and is slated for destruction
in the mythical Python 3000.%
\footnote{Python 3000 is a future python release that will clean up several
things that Guido considers to be warts.%
} Since default integer division will be removed in the future, you
can invoke the time machine with the \texttt{from \_\_future\_\_}
directives; these directives allow python programmers today to use
features that will become standard in future releases but are not
included by default because they would break existing code. From future
directives should be among the first lines you type in your python
code if you are going to use them, otherwise they may not work. The
future division operator will assume floating point division by default,%
\footnote{You may have noticed that 2/3 was represented as 0.66666666666666663
and not 0.66666666666666666 as might be expected. This is because
computers are binary calculators, and there is no exact binary representation
of 2/3, just as there is no exact binary representation of 0.1

\begin{lyxcode}
>\,{}>\,{}>~0.1

0.10000000000000001
\end{lyxcode}
Some languages try and hide this from you, but python is explicit.%
}and provides another operator // to do classic integer division.

\begin{lyxcode}
>\,{}>\,{}>~from~\_\_future\_\_~import~division

>\,{}>\,{}>~2/3

0.66666666666666663

>\,{}>\,{}>~2//3

0
\end{lyxcode}
python has four basic numeric types: int, long, float and complex,
but unlike C++, BASIC, FORTRAN or Java, you don't have to declare
these types. python can infer them

\begin{lyxcode}
>\,{}>\,{}>~type(1)

<type~'int'>

>\,{}>\,{}>~type(1.0)

<type~'float'>

>\,{}>\,{}>~type(2{*}{*}200)

<type~'long'>


\end{lyxcode}
$2^{200}$is a huge number!

\begin{lyxcode}
>\,{}>\,{}>~2{*}{*}200

1606938044258990275541962092341162602522202993782792835301376L
\end{lyxcode}
but python will blithely compute it and much larger numbers for you
as long as you have CPU and memory to handle them. The integer type,
if it overflows, will automatically convert to a python \texttt{long}
(as indicated by the appended \texttt{L} in the output above) and
has no built-in upper bound on size, unlike C/C++ longs.

Python has built in support for complex numbers. Eg, we can verify
$i^{2}=-1$ 

\begin{lyxcode}
>\,{}>\,{}>~x~=~complex(0,1)

>\,{}>\,{}>~x{*}x

(-1+0j)
\end{lyxcode}
To access the real and imaginary parts of a complex number, use the
\texttt{real} and \texttt{imag} attributes

\begin{lyxcode}
>\,{}>\,{}>~x.real

0.0

>\,{}>\,{}>~x.imag

1.0
\end{lyxcode}
If you come from other languages like Matlab, the above may be new
to you. In matlab, you might do something like this (>\,{}> is the
standard matlab shell prompt)

\begin{lyxcode}
>\,{}>~x~=~0+j

x~=

~~~0.0000~+~1.0000i



>\,{}>~real(x)

ans~=

~~~~~0



>\,{}>~imag(x)

ans~=

~~~~~1




\end{lyxcode}
That is, in Matlab, you use a \textit{function} to access the real
and imaginary parts of the data, but in python these are attributes
of the complex object itself. This is a core feature of python and
other object oriented languages: an object carries its data and methods
around with it. One might say: {}``a complex number knows it's real
and imaginary parts'' or {}``a complex number knows how to take
its conjugate'', you don't need external functions for these operations

\begin{lyxcode}
>\,{}>\,{}>~x.conjugate

<built-in~method~conjugate~of~complex~object~at~0xb6a62368>

>\,{}>\,{}>~x.conjugate()

-1j
\end{lyxcode}
On the first line, I just followed along from the example above with
\texttt{real} and \texttt{imag} and typed \texttt{x.conjugate} and
python printed the representation \texttt{<built-in method conjugate
of complex object at 0xb6a62368>.} This means that \texttt{conjugate}
is a \textit{method}, a.k.a a function, and in python we need to use
parentheses to call a function. If the method has arguments, like
the \texttt{x} in \texttt{sin(x)}, you place them inside the parentheses,
and if it has no arguments, like \texttt{conjugate}, you simply provide
the open and closing parentheses. \texttt{real}, \texttt{imag} and
\texttt{conjugate} are attributes of the complex object, and \texttt{conjugate}
is a \textit{callable} attribute, known as a \textit{method}.

OK, now you are an object oriented programmer. There are several key
ideas in object oriented programming, and this is one of them: an
object carries around with it data (simple attributes) and methods
(callable attributes) that provide additional information about the
object and perform services. It's one stop shopping -- no need to
go to external functions and libraries to deal with it -- the object
knows how to deal with itself.


\section[Standard Library]{Accessing the standard library}

Arithmetic is fine, but before long you may find yourself tiring of
it and wanting to compute logarithms and exponents, sines and cosines

\begin{lyxcode}
>\,{}>\,{}>~log(10)

Traceback~(most~recent~call~last):

~~File~\char`\"{}<stdin>\char`\"{},~line~1,~in~?

NameError:~name~'log'~is~not~defined
\end{lyxcode}
These functions are not built into python, but don't despair, they
are built into the python standard library. To access a function from
the standard library, or an external library for that matter, you
must import it.

\begin{lyxcode}
>\,{}>\,{}>~import~math

>\,{}>\,{}>~math.log(10)

2.3025850929940459

>\,{}>\,{}>~math.sin(math.pi)

1.2246063538223773e-16
\end{lyxcode}
Note that the default \texttt{log} function is a base 2 logarithm
(use \texttt{math.log10} for base 10 logs) and that floating point
math is inherently imprecise, since analytically$\sin(\pi)=0$.

It's kind of a pain to keep typing \texttt{math.log} and \texttt{math.sin}
and \texttt{math.p}i, and python is accomodating. There are additional
forms of \texttt{import} that will let you save more or less typing
depending on your desires

\begin{lyxcode}
\textcolor{blue}{\#~Appreviate~the~module~name:~m~is~an~alias}

>\,{}>\,{}>~import~math~as~m

>\,{}>\,{}>~m.cos(2{*}m.pi)

1.0



\textcolor{blue}{\#~Import~just~the~names~you~need}

>\,{}>\,{}>~from~math~import~exp,~log

>\,{}>\,{}>~log(exp(1))

1.0



\textcolor{blue}{\#~Import~everything~-~use~with~caution!}

>\,{}>\,{}>~from~math~import~{*}

>\,{}>\,{}>~sin(2{*}pi{*}10)

-2.4492127076447545e-15
\end{lyxcode}
To help you learn more about what you can find in the math library,
python has nice introspection capabilities -- introspection is a way
of asking an object about itself. For example, to find out what is
available in the math library, we can get a directory of everything
available with the \texttt{dir} command%
\footnote{In addition to the introdpection and help provided in the python interpreter,
the official documentation of the python standard library is very
good and up-to-date http://docs.python.org/lib/lib.html .%
}

\begin{lyxcode}
>\,{}>\,{}>~dir(math)

{[}'\_\_doc\_\_',~'\_\_file\_\_',~'\_\_name\_\_',~'acos',~'asin',~'atan',~'atan2',~'ceil',~'cos',~'cosh',~'degrees',~'e',~'exp',~'fabs',~'floor',~'fmod',~'frexp',~'hypot',~'ldexp',~'log',~'log10',~'modf',~'pi',~'pow',~'radians',~'sin',~'sinh',~'sqrt',~'tan',~'tanh']
\end{lyxcode}
This gives us just a listing of the names that are in the math module
-- they are fairly self descriptive, but if you want more, you can
call \texttt{help} on any of these functions for more information

\begin{lyxcode}
>\,{}>\,{}>~help(math.sin)~

Help~on~built-in~function~sin:

sin(...)

sin(x)

Return~the~sine~of~x~(measured~in~radians).
\end{lyxcode}
and for the whole math library

\begin{lyxcode}
>\,{}>\,{}>~help(math)~

Help~on~module~math:

~

NAME

~~~~math

~

FILE

~~~~/usr/local/lib/python2.3/lib-dynload/math.so

~

DESCRIPTION

~~~~This~module~is~always~available.~~It~provides~access~to~the

~~~~mathematical~functions~defined~by~the~C~standard.

~

FUNCTIONS

~~~~acos(...)

~~~~~~~~acos(x)

~~~~~~~~~

~~~~~~~~Return~the~arc~cosine~(measured~in~radians)~of~x.

~~~~~

~~~~asin(...)

~~~~~~~~asin(x)

~~~~~~~~~

~~~~~~~~Return~the~arc~sine~(measured~in~radians)~of~x.

~~~~~
\end{lyxcode}
And much more which is snipped. Likewise, we can get information on
the complex object in the same way

\begin{lyxcode}
>\,{}>\,{}>~x~=~complex(0,1)

>\,{}>\,{}>~dir(x)

{[}'\_\_abs\_\_',~'\_\_add\_\_',~'\_\_class\_\_',~'\_\_coerce\_\_',~'\_\_delattr\_\_',~'\_\_div\_\_',~'\_\_divmod\_\_',~'\_\_doc\_\_',~'\_\_eq\_\_',~'\_\_float\_\_',~'\_\_floordiv\_\_',~'\_\_ge\_\_',~'\_\_getattribute\_\_',~'\_\_getnewargs\_\_',~'\_\_gt\_\_',~'\_\_hash\_\_',~'\_\_init\_\_',~'\_\_int\_\_',~'\_\_le\_\_',~'\_\_long\_\_',~'\_\_lt\_\_',~'\_\_mod\_\_',~'\_\_mul\_\_',~'\_\_ne\_\_',~'\_\_neg\_\_',~'\_\_new\_\_',~'\_\_nonzero\_\_',~'\_\_pos\_\_',~'\_\_pow\_\_',~'\_\_radd\_\_',~'\_\_rdiv\_\_',~'\_\_rdivmod\_\_',~'\_\_reduce\_\_',~'\_\_reduce\_ex\_\_',~'\_\_repr\_\_',~'\_\_rfloordiv\_\_',~'\_\_rmod\_\_',~'\_\_rmul\_\_',~'\_\_rpow\_\_',~'\_\_rsub\_\_',~'\_\_rtruediv\_\_',~'\_\_setattr\_\_',~'\_\_str\_\_',~'\_\_sub\_\_',~'\_\_truediv\_\_',~'conjugate',~'imag',~'real']


\end{lyxcode}
Notice that called \texttt{dir} or \texttt{help} on the \texttt{math}
\textit{module}, the \texttt{math.sin} \textit{function}, and the
\texttt{complex} \textit{number} \texttt{x}. That's because modules,
functions and numbers are all \textit{objects}, and we use the same
object introspection and help capabilites on them. We can find out
what type of object they are by calling \texttt{type} on them, which
is another function in python's introspection arsenal

\begin{lyxcode}
>\,{}>\,{}>~type(math)

<type~'module'>

>\,{}>\,{}>~type(math.sin)

<type~'builtin\_function\_or\_method'>

>\,{}>\,{}>~type(x)

<type~'complex'>


\end{lyxcode}
Now, you may be wondering: what were all those god-awful looking double
underscore methods, like \texttt{\_\_abs\_\_} and \texttt{\_\_mul\_\_}
in the \texttt{dir} listing of the complex object above? These are
methods that define what it means to be a numeric type in python,
and the complex object implements these methods so that complex numbers
act like the way should, eg \texttt{\_\_mul\_\_} implements the rules
of complex multiplication. The nice thing about this is that python
specifies an application programming interface (API) that is the definition
of what it means to be a number in python. And this means you can
define your own numeric types, as long as you implement the required
special double underscore methods for your custom type. double underscore
methods are very important in python; although the typical newbie
never sees them or thinks about them, they are there under the hood
providing all the python magic, and more importantly, showing the
way to let you make magic.


\section{\label{sec:intro_string}Strings}

We've encountered a number of types of objects above: int, float,
long, complex, method/function and module. We'll continue our tour
with an introduction to strings, which are critical components of
almost every program. You can create strings in a number of different
ways, with single quotes, double quotes, or triple quotes -- this
diversity of methods makes it easy if you need to embed string characters
in the string itself

\begin{lyxcode}
\textcolor{blue}{\#~single,~double~and~triple~quoted~strings}

>\,{}>\,{}>~s~=~'Hi~Mom!'

>\,{}>\,{}>~s~=~\char`\"{}Hi~Mom!\char`\"{}

>\,{}>\,{}>~s~=~\char`\"{}\char`\"{}\char`\"{}Porky~said,~\char`\"{}That's~all~folks!\char`\"{}~\char`\"{}\char`\"{}\char`\"{}
\end{lyxcode}
You can add strings together to concatenate them

\begin{lyxcode}
\textcolor{blue}{\#~concatenating~strings}

>\,{}>\,{}>~first~=~'John'

>\,{}>\,{}>~last~=~'Hunter'

>\,{}>\,{}>~first+last

'JohnHunter'
\end{lyxcode}
or call string methods to process them: upcase them or downcase them,
or replace one character with another

\begin{lyxcode}
\textcolor{blue}{\#~string~methods}

>\,{}>\,{}>~last.lower()

'hunter'

>\,{}>\,{}>~last.upper()

'HUNTER'

>\,{}>\,{}>~last.replace('h',~'p')

'Hunter'

>\,{}>\,{}>~last.replace('H',~'P')

'Punter'~
\end{lyxcode}
Note that in all of these examples, the string \texttt{last} is unchanged.
All of these methods operate on the string and return a new string,
leaving the original unchanged. In fact, python strings cannot be
changed by any python code at all: they are \textit{immutable} (unchangeable).
The concept of mutable and immutable objects in python is an important
one, and it will come up again, because only immutable objects can
be used as keys in python dictionaries and elements of python sets.

You can access individual characters, or slices of the string (substrings),
using indexing. A string in sequence of characters, and strings implement
the sequence protocol in python -- we'll see more examples of python
sequences later -- and all sequences have the same syntax for accessing
their elements. Python uses 0 based indexing which means the first
element is at index 0; you can use negative indices to access the
last elements in the sequence

\begin{lyxcode}
\textcolor{blue}{\#~string~indexing}

>\,{}>\,{}>~last~=~'Hunter'

>\,{}>\,{}>~last{[}0]

'H'

>\,{}>\,{}>~last{[}1]

'u'

>\,{}>\,{}>~last{[}-1]~

'r'~
\end{lyxcode}
To access substrings, or generically in terms of the sequence protocol,
slices, you use a colon to indicate a range

\begin{lyxcode}
\textcolor{blue}{\#~string~slicing}

>\,{}>\,{}>~last{[}0:2]

'Hu'

>\,{}>\,{}>~last{[}2:4]

'nt'
\end{lyxcode}
As this example shows, python uses {}``one-past-the-end'' indexing
when defining a range; eg, in the range \texttt{indmin:indmax}, the
element of \texttt{imax} is not included. You can use negative indices
when slicing too; eg, to get everything before the last character

\begin{lyxcode}
>\,{}>\,{}>~last{[}0:-1]

'Hunte'
\end{lyxcode}
You can also leave out either the min or max indicator; if they are
left out, 0 is assumed to be the \texttt{indmin} and one past the
end of the sequence is assumed to be \texttt{indmax}

\begin{lyxcode}
>\,{}>\,{}>~last{[}:3]

'Hun'

>\,{}>\,{}>~last{[}3:]

'ter'
\end{lyxcode}
There is a third number that can be placed in a slice, a step, with
syntax indmin:indmax:step; eg, a step of 2 will skip every second
letter

\begin{lyxcode}
>\,{}>\,{}>~last{[}1:6:2]

'utr'
\end{lyxcode}
Although this may be more that you want to know about slicing strings,
the time spent here is worthwhile. As mentioned above, all python
sequences obey these rules. In addition to strings, lists and tuples,
which are built-in python sequence data types and are discussed in
the next section, the numeric arrays widely used in scientific computing
also implement the sequence protocol, and thus have the same slicing
rules.

\begin{xca}
What would you expect last{[}:] to return?
\end{xca}
One thing that comes up all the time is the need to create strings
out of other strings and numbers, eg to create filenames from a combination
of a base directory, some base filename, and some numbers. Scientists
like to create lots of data files like and then write code to loop
over these files and analyze them. We're going to show how to do that,
starting with the newbie way and progressively building up to the
way of python zen master. All of the methods below \textit{work},
but the zen master way will more efficient, more scalable (eg to larger
numbers of files) and cross-platform.%
\footnote{{}``But it works'' is a common defense of bad code; my rejoinder
to this is {}``A computer scientist is someone who fixes things that
aren't broken''. %
} Here's the newbie way: we also introduce the for-loop here in the
spirit of diving into python -- note that python uses whitespace indentation
to delimit the for-loop code block

\begin{lyxcode}
\textcolor{blue}{\#~The~newbie~way}

for~i~in~(1,2,3,4):

~~~~fname~=~'data/myexp0'~+~str(i)~+~'.dat'

~~~~print~fname
\end{lyxcode}
Now as promised, this will print out the 4 file names above, but it
has three flaws: it doesn't scale to 10 or more files, it is inefficient,
and it is not cross platform. It doesn't scale because it hard-codes
the '\texttt{0}' after \texttt{myexp}, it is inefficient because to
add several strings requires the creation of temporary strings, and
it is not cross-platform because it hard-codes the directory separator
'/'.

\begin{lyxcode}
\textcolor{blue}{\#~On~the~path~to~elightenment}

for~i~in~(1,2,3,4):

~~~~fname~=~'data/myexp\%02d.dat'\%i

~~~~print~fname
\end{lyxcode}
This example uses string interpolation, the funny \% thing. If you
are familiar with C programming, this will be no surprise to you (on
linux/unix systems do \texttt{man sprintf} at the unix shell). The
percent character is a string formatting character: \texttt{\%02d}
means to take an integer (the \texttt{d} part) and print it with two
digits, padding zero on the left (the \texttt{\%02} part). There is
more to be said about string interpolation, but let's finish the job
at hand. This example is better than the newbie way because is scales
up to files numbered 0-99, and it is more efficient because it avoids
the creation of temporary strings. For the platform independent part,
we go to the python standard library \texttt{os.path}, which provides
a host of functions for platform-independent manipulations of filenames,
extensions and paths. Here we use \texttt{os.path.join} to combine
the directory with the filename in a platform independent way. On
windows, it will use the windows path separator '\textbackslash{}'
and on unix it will use '/'.

\begin{lyxcode}
\textcolor{blue}{\#~the~zen~master~approach}

import~os

for~i~in~(1,2,3,4):

~~~~fname~=~os.path.join('data',~'myexp\%02d.dat'\%i)

~~~~print~fname
\end{lyxcode}
\begin{xca}
Suppose you have data files named like
\end{xca}
\begin{lyxcode}
data/2005/exp0100.dat

data/2005/exp0101.dat

data/2005/exp0102.dat

...

data/2005/exp1000.dat
\end{lyxcode}
Write the python code that iterates over these files, constructing
the filenames as strings in using \texttt{os.path.join} to construct
the paths in a platform-independent way. \textit{Hint}: read the help
for \texttt{os.path.join}!

OK, I promised to torture you a bit more with string interpolation
-- don't worry, I remembered. The ability to properly format your
data when printing it is crucial in scientific endeavors: how many
signficant digits do you want, do you want to use integer, floating
point representation or exponential notation? These three choices
are provided with \texttt{\%d}, \texttt{\%f} and \texttt{\%e}, with
lots of variations on the theme to indicate precision and more

\begin{lyxcode}
>\,{}>\,{}>~'warm~for~\%d~minutes~at~\%1.1f~C'~\%~(30,~37.5)

'warm~for~30~minutes~at~37.5~C'



>\,{}>\,{}>~'The~mass~of~the~sun~is~\%1.4e~kg'\%~(1.98892{*}10{*}{*}30)

'The~mass~of~the~sun~is~1.9889e+30~kg'


\end{lyxcode}
There are two string methods, \texttt{split} and \texttt{join}, that arise
frequenctly in numerical processing, specifically in the context of processing
data files that have comma, tab, or space separated numbers in
them. \texttt{split} takes a single string, and splits it on the indicated
character to a sequence of strings. This is useful to take a single line of
space or comma separated values and split them into individual numbers

\begin{lyxcode}
\textcolor{blue}{\#~s~is~a~single~string~and~we~split~it~into~a~list~of~strings}

\textcolor{blue}{\#~for~further~processing}

>\,{}>\,{}>~s~=~'1.0~2.0~3.0~4.0~5.0'

>\,{}>\,{}>~s.split('~')

{[}'1.0',~'2.0',~'3.0',~'4.0',~'5.0']
\end{lyxcode}
The return value, with square brackets, indicates that python has
returned a list of strings. These individual strings need further
processing to convert them into actual floats, but that is the first
step.  The conversion to floats will be discussed in the next session,
when we learn about list comprehensions. The converse method is join,
which is often used to create string output to an ASCII file from
a list of numbers. In this case you want to join a list of numbers
into a single line for printing to a file. The example below will
be clearer after the next section, in which lists are discussed

\begin{lyxcode}
\textcolor{blue}{\#~vals~is~a~list~of~floats~and~we~convert~it~to~a~single}

\textcolor{blue}{\#~space~separated~string}

>\,{}>\,{}>~vals~=~{[}1.0,~2.0,~3.0,~4.0,~5.0]

>\,{}>\,{}>~'~'.join({[}str(val)~for~val~in~vals])

'1.0~2.0~3.0~4.0~5.0'
\end{lyxcode}
There are two new things in the example above. One, we called the
join method directly on a string itself, and not on a variable name.
Eg, in the previous examples, we always used the name of the object
when accessing attributes, eg \texttt{x.real} or \texttt{s.upper()}.
In this example, we call the \texttt{join} method on the string which
is a single space. The second new feature is that we use a list comprehension
\texttt{{[}str(val) for val in vals]} as the argument to \texttt{join}.
\texttt{join} requires a sequence of strings, and the list comprehension
converts a list of floats to a strings. This can be confusing at first,
so don't dispair if it is. But it is worth bringing up early because
list comprehensions are a very useful feature of python. To help elucidate,
compare \texttt{vals}, which is a list of floats, with the conversion
of \texttt{vals} to a list of strings using list comprehensions in
the next line

\begin{lyxcode}
\textcolor{blue}{\#~converting~a~list~of~floats~to~a~list~of~strings}

>\,{}>\,{}>~vals

{[}1.0,~2.0,~3.0,~4.0,~5.0]

>\,{}>\,{}>~{[}str(val)~for~val~in~vals]~

{[}'1.0',~'2.0',~'3.0',~'4.0',~'5.0']
\end{lyxcode}

\section[Data Structures]{The basic python data structures}

Strings, covered in the last section, are sequences of characters.
python has two additional built-in sequence types which can hold arbitrary
elements: tuples and lists. tuples are created using parentheses,
and lists are created using square brackets

\begin{lyxcode}
\textcolor{blue}{\#~a~tuple~and~a~list~of~elements~of~the~same~type}

\textcolor{blue}{\#~(homogeneous)}

>\,{}>\,{}>~t~=~(1,2,3,4)~~\#~tuple

>\,{}>\,{}>~l~=~{[}1,2,3,4]~~\#~list
\end{lyxcode}
Both tuples and lists can also be used to hold elements of different
types

\begin{lyxcode}
\textcolor{blue}{\#~a~tuple~and~list~of~int,~string,~float}

>\,{}>\,{}>~t~=~(1,'john',~3.0)

>\,{}>\,{}>~l~=~{[}1,'john',~3.0]
\end{lyxcode}
Tuples and lists have the same indexing and slicing rules as each
other, and as string discussed above, because both implement the python
sequence protocol, with the only difference being that tuple slices
return tuples (indicated by the parentheses below) and list slices
return lists (indicated by the square brackets)

\begin{lyxcode}
\#~indexing~and~slicing~tuples~and~lists

>\,{}>\,{}>~t{[}0]

1

>\,{}>\,{}>~l{[}0]

1

>\,{}>\,{}>~t{[}:-1]

(1,~'john')

>\,{}>\,{}>~l{[}:-1]

{[}1,~'john']
\end{lyxcode}
So why the difference between tuples and lists? A number of explanations
have been offered on the mailing lists, but the only one that makes
a difference to me is that tuples are immutable, like strings, and
hence can be used as keys to python dictionaries and included as elements
of sets, and lists are mutable, and cannot. So a tuple, once created,
can never be changed, but a list can. For example, if we try to reassign
the first element of the tuple above, we get an error

\begin{lyxcode}
>\,{}>\,{}>~t{[}0]~=~'why~not?'

Traceback~(most~recent~call~last):

~File~\char`\"{}<stdin>\char`\"{},~line~1,~in~?

TypeError:~object~doesn't~support~item~assignment
\end{lyxcode}
But the same operation is perfectly accetable for lists

\begin{lyxcode}
>\,{}>\,{}>~l{[}0]~=~'why~not?'

>\,{}>\,{}>~l

{[}'why~not?',~'john',~3.0]
\end{lyxcode}
lists also have a lot of methods, tuples have none, save the special
double underscore methods that are required for python objects and
sequences

\begin{lyxcode}
\textcolor{blue}{\#~tuples~contain~only~{}``hidden''~double~underscore~methods}

>\,{}>\,{}>~dir(t)

{[}'\_\_add\_\_',~'\_\_class\_\_',~'\_\_contains\_\_',~'\_\_delattr\_\_',~'\_\_doc\_\_',~'\_\_eq\_\_',~'\_\_ge\_\_',~'\_\_getattribute\_\_',~'\_\_getitem\_\_',~'\_\_getnewargs\_\_',~'\_\_getslice\_\_',~'\_\_gt\_\_',~'\_\_hash\_\_',~'\_\_init\_\_',~'\_\_iter\_\_',~'\_\_le\_\_',~'\_\_len\_\_',~'\_\_lt\_\_',~'\_\_mul\_\_',~'\_\_ne\_\_',~'\_\_new\_\_',~'\_\_reduce\_\_',~'\_\_reduce\_ex\_\_',~'\_\_repr\_\_',~'\_\_rmul\_\_',~'\_\_setattr\_\_',~'\_\_str\_\_']



\textcolor{blue}{\#~but~lists~contain~other~methods,~eg~append,~extend~and}

\textcolor{blue}{\#~reverse}

>\,{}>\,{}>~dir(l)

{[}'\_\_add\_\_',~'\_\_class\_\_',~'\_\_contains\_\_',~'\_\_delattr\_\_',~'\_\_delitem\_\_',~'\_\_delslice\_\_',~'\_\_doc\_\_',~'\_\_eq\_\_',~'\_\_ge\_\_',~'\_\_getattribute\_\_',~'\_\_getitem\_\_',~'\_\_getslice\_\_',~'\_\_gt\_\_',~'\_\_hash\_\_',~'\_\_iadd\_\_',~'\_\_imul\_\_',~'\_\_init\_\_',~'\_\_iter\_\_',~'\_\_le\_\_',~'\_\_len\_\_',~'\_\_lt\_\_',~'\_\_mul\_\_',~'\_\_ne\_\_',~'\_\_new\_\_',~'\_\_reduce\_\_',~'\_\_reduce\_ex\_\_',~'\_\_repr\_\_',~'\_\_rmul\_\_',~'\_\_setattr\_\_',~'\_\_setitem\_\_',~'\_\_setslice\_\_',~'\_\_str\_\_',~'append',~'count',~'extend',~'index',~'insert',~'pop',~'remove',~'reverse',~'sort']
\end{lyxcode}
Many of these list methods change, or mutate, the list, eg append
adds an element to the list\texttt{: extend} extends the list with
a sequence of elements, \texttt{sort} sorts the list in place, \texttt{reverse}
reverses it in place, \texttt{pop} takes an element off the list and
returns it.

We've seen a couple of examples of creating a list above -- let's
look at some more using list methods

\begin{lyxcode}
>\,{}>\,{}>~x~=~{[}]~~~~~~~~~~~~~~~~~~~\textcolor{blue}{\#~create~the~empty~list}

>\,{}>\,{}>~x.append(1)~~~~~~~~~~~~~~\textcolor{blue}{\#~add~the~integer~one~to~it}

>\,{}>\,{}>~x.extend({[}'hi',~'mom'])~~\textcolor{blue}{\#~append~two~strings~to~it}

>\,{}>\,{}>~x

{[}1,~'hi',~'mom']

>\,{}>\,{}>~x.reverse()~~~~~~~~~~~~~~\textcolor{blue}{\#~reverse~the~list,~in~place}

>\,{}>\,{}>~x

{[}'mom',~'hi',~1]

>\,{}>\,{}>~len(x)

3
\end{lyxcode}
We mentioned list comprehensions in the last section when discussing
string methods.  List comprehensions are a way of creating a list
using a for loop in a single line of python. Let's create a list of
the perfect cubes from 1 to 10, first with a for loop and then with
a list comprehension. The list comprehension code will not only be
shorter and more elegant, it can be much faster (the dots are the
indentation block indicator from the python shell and should not be
typed)

\begin{lyxcode}
\textcolor{blue}{\#~a~list~of~perfect~cubes~using~a~for-loop}

>\,{}>\,{}>~cubes~=~{[}]

>\,{}>\,{}>~for~i~in~range(1,10):

...~~~~~cubes.append(i{*}{*}3)

...~

>\,{}>\,{}>~cubes

{[}1,~8,~27,~64,~125,~216,~343,~512,~729]



\textcolor{blue}{\#~functionally~equivalent~code~using~list~comprehensions}

>\,{}>\,{}>~cubes~=~{[}i{*}{*}3~for~i~in~range(1,10)]

>\,{}>\,{}>~cubes

{[}1,~8,~27,~64,~125,~216,~343,~512,~729]
\end{lyxcode}
The list comprehension code is faster because it all happens at the
C level.  In the simple for-loop version, the python expression which
appends the cube of \texttt{i} has to be evaluated by the python interpreter
for each element of the loop. In the list comprehension example, the
single line is parsed once and executed at the C level.  The difference
in speed can be considerable, and the list comprehension example is
shorter and more elegant to boot.

The remaining essential built-in data strucuture in python is the
dictionary, which is an associative array that maps arbitrary immutable
objects to arbitrary objects. int, long, float, string and tuple are
all immutable and can be used as keys; to a dictionary list and dict
are mutable and cannot. A dictionary takes one kind of object as the
key, and this key points to another object which is the value. In
a contrived but easy to comprehent examples, one might map names to
ages

\begin{lyxcode}
>\,{}>\,{}>~ages~=~\{\}~~~~~~~~~~~~\textcolor{blue}{\#~create~an~empty~dict}

>\,{}>\,{}>~ages{[}'john']~=~36

>\,{}>\,{}>~ages{[}'fernando']~=~33

>\,{}>\,{}>~ages~~~~~~~~~~~~~~~~~\textcolor{blue}{\#~view~the~whole~dict}

\{'john':~36,~'fernando':~33\}

>\,{}>\,{}>~ages{[}'john']

36

>\,{}>\,{}>~ages{[}'john']~=~37~~~~\textcolor{blue}{\#~reassign~john's~age}

>\,{}>\,{}>~ages{[}'john']

37
\end{lyxcode}
Dictionary lookup is very fast; Tim Peter's once joked that any python
program which uses a dictionary is automatically 10 times faster than
any C program, which is of course false, but makes two worthy points
in jest: dictionary lookup is fast, and dictionaries can be used for
important optimizations, eg, creating a cache of frequently used values.
As a simple eaxample, suppose you needed to compute the product of
two numbers between 1 and 100 in an inner loop -- you could use a
dictionary to cache the cube of all odd of numbers < 100; if you were
inteterested in all numbers, you might simply use a list to store
the cached cubes -- I am cacheing only the odd numbers to show you
how a dictionary can be used to represent a sparse data structure

\begin{lyxcode}


>\,{}>\,{}>~cubes~=~dict({[}~(~i,~i{*}{*}3~)~for~i~in~range(1,100,2)])

>\,{}>\,{}>~cubes{[}5]

125
\end{lyxcode}
The last example is syntactically a bit challenging, but bears careful
study.  We are initializing a dictionary with a list comprehension.
 The list comprehension is made up of length 2 tuples \texttt{( i,
i{*}{*}3} ).  When a dictionary is initialized with a sequence of
length 2 tuples, it assumes the first element of the tuple \texttt{i}
is the \textit{key} and the second element i{*}{*}3is the \textit{value}.
 Thus we have a lookup table from odd integers to to cube.  Creating
dictionaries from list comprehensions as in this example is something
that hard-core python programmers do almost every day, and you should
too.

\begin{xca}
Create a lookup table of the product of all pairs of numbers less
than 100. The key will be a tuple of the two numbers \texttt{(i,j)}
and the value will be the product. Hint: you can loop over multiple
ranges in a list comprehension, eg \texttt{{[} something for i in
range(Ni) for j in range(Nj)]}
\end{xca}

\section[Zen]{The Zen of Python}

\begin{xca}
\texttt{>\,{}>\,{}> import this}
\end{xca}

\section{Functions and classes}

You can define functions just about anywhere in python code. The typical
function definition takes zero or more arguments, zero or more keyword
arguments, and is followed by a documentation string and the function
definition, optionally returing a value. Here is a function to compute
the hypoteneuse of a right triange

\begin{lyxcode}
def~hypot(base,~height):

~~~'compute~the~hypoteneuse~of~a~right~triangle'

~~~import~math

~~~return~math.sqrt(base{*}{*}2~+~height{*}{*}2)
\end{lyxcode}
As in the case of the for-loop, leading white space is significant
and is used to delimt the start and end of the function. In the example
below, x = 1 is not in the function, because it is not indented

\begin{lyxcode}
def~growone(l):

~~~'append~1~to~a~list~l'

~~~l.append(1)

x~=~1
\end{lyxcode}
Note that this function does not return anything, because the append
method modifies the list that was passed in. You should be careful
when designing functions that have side effects such as modifying
the structures that are passed in; they should be named and documented
in such a way that these side effects are clear.

Python is pretty flexible with functions: you can define functions
within function definitions (just be mindful of your indentation),
you can attach attributes to functions (like other objects), you can
pass functions as arguments to other functions. A function keyword
argument defines a default value for a function that can be overridden.
Below is an example which provides a normalize keyword argument. The
default argument is \texttt{normalize=None}; the value None is a standard
python idiom which usually means either do the default thing or do
nothing. If \texttt{normalize} is not \texttt{None}, we assume it
is a function that can be called to normalize our data

\begin{lyxcode}
def~psd(x,~normalize=None):

~~~~'compute~the~power~spectral~density~of~x'

~~~~if~normalize~is~not~None:~x~=~normalize(x)

~~~\textcolor{blue}{~\#~compute~the~power~spectra~of~x~and~return~it}
\end{lyxcode}
This function could be called with or without a \texttt{normalize}
keyword argument, since if the argument is not passed, the default
of \texttt{None} is used and no normalization is done.

\begin{lyxcode}


\textcolor{blue}{\#~no~normalize~argument;~do~the~default~thing}

>\,{}>\,{}>~psd(x)~~~



\textcolor{blue}{\#~define~a~custom~normalize~function~unitstd~as~pass~it}

\textcolor{blue}{\#~to~psd}

>\,{}>\,{}>~def~unitstd(x):~return~x/std(x)

>\,{}>\,{}>~psd(x,~normalize=unitstd)


\end{lyxcode}
In Section\ref{sec:into_calculator} we noticed that complex objects
have the real and imag data attributes, and the conjugate method.
An object is an instance of a class that defines it, and in python
you can easily define your own classes. In that section, we emphasized
that one of the important features of a classes/objects is that they
carry around their data and methods in a single bundle. Let's look
at the mechnics of defining classes, and creating instances (a.k.a.
objects) of these classes. Classes have a special double underscore
method \_\_init\_\_ that is used as the function to initialize the
class. For this example, we'll continue with the normalize theme above,
but in this case the normalization requires some data parameters.
This example arises when you want to normalize an image which may
range over 0-255 (8 bit image) or from 0-65535 (16 bit image) to the
0-1 interval. For 16 bit images, you would normally divide everything
by 65525, but you might want to configure this to a smaller number
if your data doesn't use the whole intensity range to enhance contrast.
For simplicitly, let's suppose our normalize class is only interested
in the pixel maximum, and will divide all the data by that value.

\begin{lyxcode}
from~\_\_future\_\_~import~division~~\textcolor{blue}{\#~make~sure~we~do~float~division}

class~Normalize:

~~~~\char`\"{}\char`\"{}\char`\"{}

~~~~A~class~to~normalize~data~by~dividing~it~by~a~maximum~value

~~~~\char`\"{}\char`\"{}\char`\"{}

~~~~def~\_\_init\_\_(self,~maxval):

~~~~~~~~'maxval~will~be~mapped~to~1'

~~~~~~~~self.maxval~=~maxval

~~~~def~\_\_call\_\_(self,~data):

~~~~~~~~'do~the~normalization'

~~~~~~~~\textcolor{blue}{\#~in~real~life~you~would~also~want~to~clip~all~values~of}

~\textcolor{blue}{~~~~~~~\#~data>maxval~so~that~the~returned~value~will~be~in~the~unit}

~\textcolor{blue}{~~~~~~~\#~interval}

~~~~~~~~return~data/self.maxval
\end{lyxcode}
The triple quoted string following the definition of class Normalize
is the class documentation stringd, and it will bre shown to the user
when they do \texttt{help(Normalize)}. A commonly used convention
is to name classes with \textit{UpperCase}, but this is not required.
self is a special variable that a class can use to refer to its own
data and methods, and must be the first argument to all the class
methods. The \texttt{\_\_init\_\_} method stores the normalization
value maxval as a class attribute in \texttt{self.maxval}, and this
value can later be reused by other class methods (as it is in \texttt{\_\_call\_\_})
and it can be altered by the user of the class, as will illustrate
below. The \texttt{\_\_call\_\_} method is another piece of python
double underscore magic, it allows class instances to be used as \textit{functions},
eg you can call them just like you can call any function. OK, now
let's see how you could use this. 

The first line use used to create an \textit{instance} of the \textit{class}
\texttt{Normalize}, and the special method \texttt{\_\_init\_\_} is
implicitly called. The second line implicitly calls the special \texttt{\_\_call\_\_}method

\begin{lyxcode}
>\,{}>\,{}>~norm~=~Normalize(65356)~\textcolor{blue}{\#~good~for~16~bit~images}

>\,{}>\,{}>~norm(255)~~~~~~~~~~~~~~~\textcolor{blue}{\#~call~this~function}

0.0039017075708427688



\textcolor{blue}{\#~We~can~reset~the~maxval~attribute,~and~the~call~method~}

\textcolor{blue}{\#~is~automagically~updated}

>\,{}>\,{}>~norm.maxval~=~255~~~~~~~\textcolor{blue}{\#~reset~the~maxval}

>\,{}>\,{}>~norm(255)~~~~~~~~~~~~~~~\textcolor{blue}{\#~and~call~it~again}

1.0



\textcolor{blue}{\#~We~can~pass~the~norm~instance~to~the~psd~function~we~defined~above,~which~}

\textcolor{blue}{\#~is~expecting~a~function}

>\,{}>\,{}>~pdf(X,~normalize=norm)~~~~~~~~~~~~
\end{lyxcode}
\begin{xca}
Pretend that \texttt{complex} were not built-in to the python core,
and write your own complex class \texttt{MyComplex}. Provide \texttt{real}
and \texttt{imag} attributes and the \texttt{conjugate} method. Define
\texttt{\_\_abs\_\_}, \texttt{\_\_mul\_\_} and \texttt{\_\_add\_\_}
to implement the absolute value of complex numbers, multiplication
of complex numbers and addition of complex numbers. See the API definition
of the python number protocol; although this is written for C programmers,
it contains information about the required function call signatures
for each of the double underscore methods that define the number protocol
in python; where they use \texttt{o1} on that page, you would use
\texttt{self} in python, and where they use \texttt{o2} you might
use \texttt{other} in python.%
\footnote{http://www.python.org/doc/current/api/number.html%
} To get you started, I'll show you what the \texttt{\_\_add\_\_} method
should look like
\end{xca}
\begin{lyxcode}
\textcolor{blue}{\#~An~example~double~underscore~method~required~in~your~MyComplex}

\textcolor{blue}{\#~implementation}

def~\_\_add\_\_(self,~other):

~~~~'add~self~to~other~and~return~a~new~MyComplex~instance'

~~~~r~=~self.real~+~other.real

~~~~i~=~self.imag~+~other.imag

~~~~return~MyComplex(r,i)



\textcolor{blue}{\#~When~you~are~finished,~test~your~implementation~with~}

>\,{}>\,{}>~x~=~MyComplex(2,3)

>\,{}>\,{}>~y~=~MyComplex(0,1)

>\,{}>\,{}>~x.real

2.0

>\,{}>\,{}>~y.imag

1.0

>\,{}>\,{}>~x.conjugate()

(2-3j)

>\,{}>\,{}>~x+y

(2+4j)

>\,{}>\,{}>~x{*}y

(-3+2j)

>\,{}>\,{}>~abs(x{*}y)

3.6055512754639891


\end{lyxcode}

\section[Files]{Files and file like objects}

Working with files is one of the most common and important things
we do in scientific computing because that is usually where the data
lives. In Section\ref{sec:intro_string}, we went through the mechanics
of automatically building file names like

\begin{lyxcode}
data/myexp01.dat

data/myexp02.dat

data/myexp03.dat

data/myexp04.dat
\end{lyxcode}
but we didn't actually do anything with these files. Here we'll show
how to read in the data and do something with it. Python makes working
with files easy and dare I say fun. The test data set lives in \texttt{data/family.csv}
and is a standard comma separated value file that contains information
about my family: first name, last name, age, height in cm, weight
in kg and birthdate. We'll open this file and parse it -- note that
python has a standard module for parsing CSV files that is much more
sophisticated than what I am doing here. Nevertheless, it serves as
an easy to understand example that is close enough to real life that
it is worth doing. Here is what the data file looks like

\begin{lyxcode}
First,Last,Age,Weight,Height,Birthday

John,Hunter,36,175,180,1968-03-05

Miriam,Sierig,33,135,177,1971-05-04

Rahel,Hunter,7,55,134,1998-02-25

Ava,Hunter,3,45,121,2001-04-26

Clara,Hunter,0,15,55,2004-10-02
\end{lyxcode}
Here is the code to parse that file

\begin{lyxcode}
\textcolor{blue}{\#~open~the~file~for~reading}

fh~=~file('../data/family.csv',~'r')

\textcolor{blue}{\#~slurp~the~header,~splitting~on~the~comma}

headers~=~fh.readline().split(',')

\textcolor{blue}{\#~now~loop~over~the~remaining~lines~in~the~file~and~parse~them}

for~line~in~fh:

~~~~\textcolor{blue}{\#~remove~any~leading~or~trailing~white~space}

~~~~line~=~line.strip()

~~~~\textcolor{blue}{\#~split~the~line~on~the~comma~into~separate~variables}

~~~~first,~last,~age,~weight,~height,~dob~=~line.split(',')

~~~~\textcolor{blue}{\#~convert~some~of~these~strings~to~floats}

~~~~age,~weight,~height~=~{[}float(val)~for~val~in~(age,~weight,~height)]

~~~~print~first,~last,~age,~weight,~height,~dob
\end{lyxcode}
This example illustrates several interesting things. The syntax for
opening a file is \texttt{file(filename, mode)} and the \texttt{mode}
is a string like \texttt{'r'} or \texttt{'w'} that determines whether
you are opening in read or write mode. You can also read and write
binary files with \texttt{'rb'} and \texttt{'wb'}. There are more
options and you should do \texttt{help(file)} to learn about them.
We then use the file \texttt{readline} method to read in the first
line of the file. This returns a string (the line of text) and we
call the string method \texttt{split(',')} to split that string wherever
it sees a comma, and this returns a list of strings which are the
headers

\begin{lyxcode}
>\,{}>\,{}>~headers

{[}'First',~'Last',~'Age',~'Weight',~'Height',~'Birthday\textbackslash{}n']
\end{lyxcode}
The new line character \texttt{'\textbackslash{}n'} at the end of
\texttt{'Birthday\textbackslash{}n'} indicates we forgot to strip
the string of whitespace. To fix that, we should have done

\begin{lyxcode}
>\,{}>\,{}>~headers~=~fh.readline().strip().split(',')

>\,{}>\,{}>~headers

{[}'First',~'Last',~'Age',~'Weight',~'Height',~'Birthday']~
\end{lyxcode}
Notice how this works like a pipeline: \texttt{fh.readline} returns
a line of text as a string; we call the string method \texttt{strip}
which returns a string with all white space (spaces, tabs, newlines)
removed from the left and right; we then call the \texttt{split} method
on this stripped string to split it into a list of strings.

Next we start to loop over the file -- this is a nice feature of python
file handles, you can iterate over them as a sequence. We've learned
our lesson about trailing newlines, so we first strip the line with
\texttt{line = line.strip()}. The rest is string processing, splitting
the line on a comma as we did for the headers, and converting the
strings to numbers where approriate by calling f\texttt{loat(val)}
for each of \texttt{age}, \texttt{weight} and \texttt{height}. Notice
how we use list comprehensions and tuple unpacking -- the age, weight,
\texttt{height = {[}float(val) for val in (age, weight, height)]}
line, to convert several values at once.

Now that we have all this data, how mught we store it. We could store
it in a \texttt{results} list

\begin{lyxcode}
results~=~{[}]

for~line~in~fh:

~~~~\textcolor{blue}{\#~process~the~line~as~above~to~get~the~variables}

~~~~results.append(~(first,~last,~age,~weight,~height,~dob)~)





\textcolor{blue}{\#~and~later~when~we~want~to~analyze~the~data}

for~first,~last,~age,~weight,~height,~dob~in~results:

~~~~\textcolor{blue}{\#~do~something~with~the~data}
\end{lyxcode}
\begin{xca}
\texttt{zip} magic.  Python has a nice funcion \texttt{zip} that lets
you do very useful things with lists of tuples.  \texttt{results}
above is a list of tuples -- each tuple is the \texttt{first}, \texttt{last},
\texttt{age}, \texttt{weight}, \texttt{height}, \texttt{dob} for a
family member.  What happens if you do 
\end{xca}
\begin{lyxcode}
>\,{}>\,{}>~first,~last,~age,~weight,~height,~dob~=~zip({*}results)
\end{lyxcode}
What is \texttt{age} now?

\begin{xca}
Write a class \texttt{Person} and store the attributes \texttt{first},
\texttt{last}, \texttt{age}, \texttt{weight}, \texttt{height}, \texttt{dob}
in that class.  Add a class instance to the results list, eg
\end{xca}
\begin{lyxcode}
results.append(Person(first,~last,~age,~weight,~height,~dob))
\end{lyxcode}
Python also has a special syntax for printing to an open writable
file object

\begin{lyxcode}
\textcolor{blue}{\#~open~the~file~for~writing}

outfile~=~file('mydata.data',~'w')~

for~x,y,z~in~myresults:

~~~~print~>\,{}>~outfile,~'\%1.3f~\%1.3f~\%1.3f'\%(x,y,z)
\end{lyxcode}
Another really nice thing about file objects is that other classes
can implement the file protcol and allow you to use them as if they
were files. For example, the StringIO module in the standard library
allows you to read and write to strings as if they were files. The
urllib.urlopen function allows you to open a remove web page as a
file object. Try this

\begin{lyxcode}
\textcolor{blue}{\#~loop~over~the~lines~in~google's~html}

from~urllib~import~urlopen

for~line~in~urlopen('http://www.google.com').readlines():

~~~~print~line,
\end{lyxcode}

