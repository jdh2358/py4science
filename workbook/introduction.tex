\chapter*{Introduction}

This book is currently a work in progress, and ultimately we hope it will
evolve into an open, community-driven document developed in tandem with the
underlying tools, by the same scientists who have written them.

The book is aimed at practicing scientists, students and in general anyone who
is looking for a modern, high-level and open platform for scientific
computing.  The Python language is in the opinion of the authors the leading
candidate today for this role.

The book is broadly divided in two parts: the first is a general discussion of
the Python tools used for scientific work, with an explanatory approach.  It is
not a complete Python reference book, as there are many excellent resources for
the base language, both in print and online.  But beyond the basic language and
the NumPy book, it should serve as reasonably self-contained description of the
core libraries for common numerical tasks.

The second part is meant as a practical workbook, and the build system used to
produce the document will in the future allow users to create custom versions
with only the examples that they deem practical for any given audience.  This
workbook approach grew out of a sequence of workshops taught by the editors at
a number of research institutions and universities, and we've found it to be
extremely convenient.

The workbook is structured as a collection of problems, meant to be solved by
the reader as programming exercises.  The entire book can be compiled in one of
two forms: either with the examples in `skeleton' form, where they contain
incomplete code meant to be filled in, or with the full solution code.  This
should enable instructors to hand out the skeleton workbook at courses and
workshops, with the solutions being available as well for after the teaching is
over.

We hope that the community will continue to contribute many more examples, so
that ultimately the projects allows for the easy construction of custom
workbooks tailored to the needs of different audiences.

\begin{flushright}
John D. Hunter and Fernando P�rez, editors.

\end{flushright}
