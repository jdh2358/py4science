This exercise introduces \texttt{pyrex} to wrap a C library for
trailing statistics.

Computation of trailing windowed statistics is common in many
quantitative data driven disciplines, particularly where there is
noisy data.  Common uses of windowed statistics are the trailing
moving average, standard deviation, minumum and maximum.  Two common
use cases which pose computational challenges for python: real time
updating of trailing statistics as live data comes in, and posthoc
computation of trailing statistics over a large data array.  In the
second case, for some statistics we can use convolution and related
techniques for efficient computation, eg of the trailing 30 sample
average

\begin{lstlisting}
    numpy.convolve(x, numpy.ones(30), mode=valid')[:len(x)]
\end{lstlisting}

but for other statistics like the trailing 30 day maximum at each
point, efficient routines like convolution are of no help.

This exercise introduces \texttt{pyrex} to efficiently solve the problem of
trailing statistics over arrays as well as for a live, incoming data
stream. A pure C library, \texttt{ringbuf}, defines a circular C
buffer and attached methods for efficiently computing trailing
averages, and \texttt{pyrex} is used to provide a pythonic API on top of this
extension code.  The rigid segregation between the C library and the
python wrappers insures that the C code can be used in other projects,
be it a matlab (TM) extension or some other C library.  The goal of
the exercise is to compute the trailing statistics \textit{mean},
\textit{median}, \textit{stddev}, \textit{min} and \textit{max} using
three approaches:

\begin{itemize}
  \item with brute force using \texttt{numpy} arrays, slices and methods

  \item with python bindings to the \texttt{ringbuf} code
    \texttt{ringbuf.Ringbuf}.

  \item using a \texttt{pyrex} extension to the
    \texttt{ringbuf.runstats} code

\end{itemize}
