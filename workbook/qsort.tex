
\section{Sorting quickly with QuickSort }

Quicksort is one of the best known, and probably the simplest, fast
algorithm for sorting $n$ items. It is fast in the sense that it
requires on average $\mathcal{O}(n\log n)$ comparisons instead of
$\mathcal{O}(n^{2})$, although a naive implementation does have quadratic
worst-case behavior.

The algorithm uses a simple divide and conquer strategy, and its implementation
is naturally recursive. Its basic steps are:

\begin{enumerate}
\item Pick an element from the list, called the pivot $p$ (any choice works).
\item Select from the rest of the list those elements smaller and those
greater than the pivot, and store them in separate lists $S$ and
$G$.
\item Recursively apply the algorithm \texttt{}to $S$ and $G$. The final
result can be written as $\sigma(S)+[p]+\sigma(G)$, where $\sigma$
represents the sorting operation, $+$ indicates list concatenation
and $[p]$ is the list containing the pivot as its single element.
\end{enumerate}
The listing~\ref{code:qsort} contains a skeleton with no implementation
but with tests already written (in the form of \emph{unit tests},
as described in the introduction).

\lstinputlisting[label=code:qsort,caption={IGNORED}]{examples/qsort.py}


\subsection*{Hints}

\begin{itemize}
\item Python has no particular syntactic requirements for implementing recursion,
but it does have a maximum recursion depth. This value can be queried
via the function \texttt{sys.getrecursionlimit()}, and it can be changed
with \texttt{sys.setrecursionlimit(new\_value)}. 
\item Like in all recursive problems, don't forget to implement an exit
condition!
\item If \texttt{L} is a list, the call \texttt{len(L)} provides its length.
\end{itemize}

