\section{Newton's method}
\label{sec:quad_newton}

\textbf{Illustrates:} functions as first class objects, use of the
scipy libraries.

Consider the problem of solving for $t$ in
\begin{equation}
  \int_{o}^{t}f(s)ds=u
\end{equation} 
where $f(s)$ is a monotonically increasing function of $s$ and $u>0$.

This problem can be simply solved if seen as a root finding question. Let
\begin{equation}
g(t)=\int_{o}^{t}f(s)ds-u,
\end{equation}
then we just need to find the root for $g(t),$ which is guaranteed
to be unique given the conditions above. 

The SciPy library includes an optimization package that contains a
Newton-Raphson solver called \texttt{scipy.optimize.newton.} This
solver can optionally take a known derivative for the function whose
roots are being sought, and in this case the derivative can be trivially
computed in exact form.

For this exercise, implement the solution for the test function 
\[
f(t)=t\sin^{2}(t), 
\]
using 
\[ 
u=\frac{1}{4}. 
\]

The listing~\ref{code:quad_newton} contains a skeleton that
includes for comparison the correct numerical value.

\lstinputlisting[label=code:quad_newton,caption={IGNORED}]{problems/quad_newton.py}
