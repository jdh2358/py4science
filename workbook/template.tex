\section{Your Example}
\label{sec:your_example}

This is your introduction.  Refer to python packages like
\texttt{numpy} and \texttt{matplotlib}, as well as functions like
\texttt{n.arange}, with ``texttt''.

Refer to figures like Figure~\ref{fig:your_figure}.  Your example code
should be included like.  

\lstinputlisting[label=code:your_example,caption={IGNORED}]{problems/your_example.py}

and your figures should be saved in the \textff{fig} subdirectory with
a PNG and EPS version. You should include it like

\begin{figure}
\begin{centering}\includegraphics[width=4in]{fig/your_example}\par\end{centering}

\caption{\label{fig:your_example}Your figure caption here}
\end{figure}

If you want to include python code inline, including ipython sessions use the following environment.  I sometimes decorate ipython sessions with comments.

\begin{listing}
# use namespaces!
In [1]: import numpy as n

In [2]: x = n.random.rand(10)

In [3]: x**2
Out[3]: 
array([ 0.08694464,  0.99225328,  0.05017794,  0.44827437,  0.25938905,
        0.10617795,  0.27397649,  0.33391573,  0.27975237,  0.06808894])

In [4]: 

\end{listing}