
\chapter[External libraries]{Interfacing with external libraries}


\section{weave}

% John - should we talk to Eric Jones about letting us use parts of his weave
% docs? There's a very old set of docs for weave, I could update some of it and
% this would make an excellent section. Weave is really amazing, but badly
% under-documented and hence unknown.

Below is a listing of examples of weave use. This needs a lot of cleaning,
as some of this code is very old and doesn't actually run with current
weave. 

\lstinputlisting{examples/wrap_weave.py}


\section{swig }


\section{f2py }

This is a rough set of notes on how to use f2py. It does NOT substitute
the official manual, but is rather meant to be used alongside with
it. 

For any non-trivial poject involving f2py, one should also keep at
hand Pierre Schnizer's excellent 'A short introduction to F2PY', available
from http://fubphpc.tu-graz.ac.at/\textasciitilde{}pierre/f2py\_tutorial.tar.gz 


\subsection{Usage for the impatient }

Start by building a scratch signature file automatically from your
Fortran sources (in this case all, you can choose only those .f files
you need): 

\begin{lyxcode}
f2py~-m~MODULENAME~-h~MODULENAME.pyf~{*}.f~
\end{lyxcode}
This writes the file MODULENAME.pyf, making the best guesses it can
from the Fortran sources. It builds an interface for the module to
be accessed as 'import adap1d' from python.

You will then edit the .pyf file to fine-tune the python interface
exhibited by the resulting extension. This means for example making
unnecessary scratch areas or array dimensions hidden, or making certain
parameters be optional and take a default value.

Then, write your setup.py file using distutils, and list the .pyf
file along with the Fortran sources it is meant to wrap. f2py will
build the module for you automatically, respecting all the interface
specifications you made in the .pyf file.

This approach is ultimately far easier than trying to get all the
declarations (especially dependencies) right through Cf2py directives
in the Fortran sources. While that may seem appealing at first, experience
seems to show that it's ultimately far more time-consuming and prone
to subtle errors. Using this approach, the first f2py pass can do
the bulk of the interface writing and only fine-tuning needs to be
done manually. I would only recommend embedded Cf2py directives for
very simple problems (where it works very well).

The only drawback of this approach is that the interface and the original
Fortran source lie in different files, which need to be kept in sync.
This increases a bit the chances of forgetting to update the .pyf
file if the Fortran interface changes (adding a parameter, for example).
However, the benefit of having explicit, clear control over f2py's
behavior far outweighs this concern. 


\subsection{Choosing a default compiler }

Set the FC\_VENDOR environment variable. This will then prevent f2py
from testing all the compilers it knows about. 


\subsection{Using Cf2py directives }

For simpler cases you may choose to go the route of Cf2py directives.
Below are some tips and examples for this approach.

Here's the signature of a simple Fortran routine: \lstinputlisting[language=Fortran]{snippets/wrap_sub_sig.f}

The above is correctly handled by f2py, but it can't know what is
meant to be input/output and what the relations between the various
variables are (such as integers which are array dimensions). If we
add the following f2py directives, the generated python interface
is a lot nicer: \lstinputlisting[language=Fortran]{snippets/wrap_sub_sig_f2py.f}



Some comments on the above:

\begin{itemize}
\item The f2py directives should come immediately after the 'subroutine'
line and before the Fortran variable lines. This allows the f2py dimension
directives to override the Fortran var({*}) directives.
\item If the Fortran code uses var(N) instead of var({*}), the f2py directives
can be placed after the Fortran declarations. This mode is preferred,
as there is less redundancy overall. The result is much simpler: 
\end{itemize}
\lstinputlisting[language=Fortran]{snippets/wrap_sub_sig_f2py.f}



Since python can automatically manage memory, it is possible to hide
the need for manually passed 'work' areas. The C/python wrapper to
the underlying fortran routine will allocate the memory for the needed
work areas on the fly. This is done by specifying intent(hide,cache).
'hide' tells f2py to remove the variable from the argument list and
'cache' tells it to auto-generate it.

In cases where the allocation cost becomes a performance problem,
one can remove the 'hide' part and make it an optional argument. In
this case it will only be generated if not given. For this, the line
above should be changed to:

\begin{lyxcode}
{\small Cf2py~~~real~{*}8~dimension(2{*}mm+2),~intent(cache),~optional,~depend(mm)~::~wrk~}{\small \par}
\end{lyxcode}
Note that this should only be done after proving that the scratch
areas are causing a performance problem. The \texttt{cache} directive
causes f2py to keep cached copies of the scratch areas, so no unnecessary
mallocs should be triggered.

Since f2py relies on Numeric arrays, all dimensions can be determined
from the arrays themselves and it is not necessary to pass them explicitly.

With all this, the resulting f2py-generated docstring becomes: 

\begin{lyxcode}
phipol~-~Function~signature:

~~phi~=~phipol(j,nodes,wei,x)

Required~arguments:

~~j~:~input~int

~~nodes~:~input~rank-1~array('d')~with~bounds~(mm)

~~wei~:~input~rank-1~array('d')~with~bounds~(mm)

~~x~:~input~rank-1~array('d')~with~bounds~(nn)

Return~objects:

~~phi~:~rank-1~array('d')~with~bounds~(nn)
\end{lyxcode}

\subsection{Debugging}

For debugging, use the --debug-capi option to f2py. This causes the
extension modules to print detailed information while in operation.
In distutils, this must be passed as an option in the f2py\_options
to the Extension constructor. 


\subsection{Wrapping C codes with f2py}

Below is Pearu Peterson's (the f2py author) response to a question
about using f2py to wrap existing C codes. While SWIG provides similar
functionality and weave is perfect for inlining C, f2py seems to be
an incredibly simple and convenient tool for wrapping C libraries.

Pearu's response follows: 

For example, consider the following C file:

\begin{lyxcode}
/{*}~foo.c~{*}/

double~foo(double~{*}x,~int~n)~\{

~~int~i;

~~double~r~=~0;

~~for~(i=0;i<n;++i)

~~~~r~+=~x{[}i];

~~return~r;

\}

/{*}~EOF~foo.c~{*}/
\end{lyxcode}
To wrap the C function foo() with f2py, create the following signature
file bar.pyf: 

\begin{lyxcode}
!~-{*}-~F90~-{*}-

python~module~bar

~~interface

~~~~real{*}8~function~foo(x,n)

~~~~~~intent(c)~foo

~~~~~~real{*}8~dimension(n),intent(in)~::~x

~~~~~~integer~intent(c,hide),depend(x)~::~n~=~len(x)

~~~~end~function~foo

~~end~interface

end~python~module~bar

!~EOF~bar.pyf
\end{lyxcode}
(see usersguide for more info about intent(c)) and run

\begin{lyxcode}
~~f2py~-c~bar.pyf~foo.c
\end{lyxcode}
Finally, in Python:

\begin{lyxcode}
>\,{}>\,{}>~import~bar

>\,{}>\,{}>~bar.foo({[}1,2,3])

6.0
\end{lyxcode}

\subsection{Passing offset arrays to Fortran routines}

It is possible to pass offset arrays (like pointers to the middle
of other arrays) by using Numeric's slice notation.

The print\_dvec function below simply prints its argument as \char`\"{}print{*},'x',x\char`\"{}.
We show some examples of how it behaves with both 1 and 2-d arrays: 

\begin{lyxcode}
In~{[}3]:~x

Out{[}3]:~array({[}~2.8,~~3.4,~~4.1])

In~{[}4]:~tf.print\_dvec(x)

~n~3

~x~~2.8~~3.4~~4.1

In~{[}5]:~tf.print\_dvec~?

Type:~~~~~~~~~~~fortran

String~Form:~~~~<fortran~object~at~0x8306fe8>

Namespace:~~~~~~Currently~not~defined~in~user~session.

Docstring:

~~~~print\_dvec~-~Function~signature:

~~~~~~print\_dvec(x,{[}n])

~~~~Required~arguments:

~~~~~~x~:~input~rank-1~array('d')~with~bounds~(n)

~~~~Optional~arguments:

~~~~~~n~:=~len(x)~input~int

In~{[}6]:~tf.print\_dvec~(x{[}1])

~n~1

~x~~3.4

In~{[}7]:~tf.print\_dvec~(x{[}1:])

~n~2

~x~~3.4~~4.1

In~{[}8]:~A

Out{[}8]:

array({[}{[}~3.5,~~5.6,~~8.2],

~~~~~~~{[}~2.1,~~4.5,~~1.2],

~~~~~~~{[}~6.3,~~3.4,~~3.1]])

In~{[}9]:~tf.print\_dvec(A)

~n~9

~x~~3.5~~5.6~~8.2~~2.1~~4.5~~1.2~~6.3~~3.4~~3.1

In~{[}10]:~A

Out{[}10]:

array({[}{[}~3.5,~~5.6,~~8.2],

~~~~~~~{[}~2.1,~~4.5,~~1.2],

~~~~~~~{[}~6.3,~~3.4,~~3.1]])

In~{[}11]:~tf.print\_dvec(A{[}1:])

~n~6

~x~~2.1~~4.5~~1.2~~6.3~~3.4~~3.1

In~{[}12]:~A{[}1:]

Out{[}12]:

array({[}{[}~2.1,~~4.5,~~1.2],

~~~~~~~{[}~6.3,~~3.4,~~3.1]])

In~{[}13]:~A{[}1:,1:]

Out{[}13]:

array({[}{[}~4.5,~~1.2],

~~~~~~~{[}~3.4,~~3.1]])

In~{[}14]:~tf.print\_dvec(A{[}1:,1:])

~n~4

~x~~4.5~~1.2~~3.4~~3.1
\end{lyxcode}

\subsection{On matrix ordering and in-memory copies}

Numeric (which f2py relies on) is C-based, and therefore its arrays
are stored in row-major order. Fortran stores its arrays in column-major
order. This means that copying issues must be dealt with. Below we
reproduce some comments from Pearu on this topic given in the f2py
mailing list in June/2002: 

\begin{quote}
To avoid copying, you should create array that has internally Fortran
data ordering. This is achived, for example, by reading/creating your
data in Fortran ordering to Numeric array and then doing Numeric.transpose
on that. Every f2py generated extension module provides also function 

has\_column\_major\_storage

to check if an array is Fortran contiguous or not. If has\_column\_major\_storage(arr)
returns true then there will be no copying for the array arr if passed
to f2py generated functions (assuming that the types are proper, of
cource).

Also note that copying done by f2py generated interface is carried
out in C on the raw data and therefore it is extremely fast compared
to if you would make a copy in Python, even when using Numeric. Tests
with say 1000x1000 matrices show that there is no noticable performance
hit when copying is carried out, in fact, sometimes making a copy
may speed up things a bit -- I was quite surprised about that myself.

So, I think, you should worry about copying only if the sizes of matrices
are really large, say, larger than 5000x5000 and efficient memory
usage is relevant. The time spent for copying is negligible even for
large arrays provided that your computer has plenty of memory (>=256MB).
\end{quote}

\subsection{Distutils}

Below is an example setup.py file which generates a Python extension
module from Fortran90 sources and a .pyf interface file generated
by f2py and later fine tuned. 

\lstinputlisting{examples/wrap_f2py_setup.py}


\section{Others}

boost, pyrex, cxx


\section[Standalone applications]{Distributing standalone applications}

py2exe, mcmillan installer
